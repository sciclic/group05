% Options for packages loaded elsewhere
\PassOptionsToPackage{unicode}{hyperref}
\PassOptionsToPackage{hyphens}{url}
%
\documentclass[
]{article}
\usepackage{lmodern}
\usepackage{amssymb,amsmath}
\usepackage{ifxetex,ifluatex}
\ifnum 0\ifxetex 1\fi\ifluatex 1\fi=0 % if pdftex
  \usepackage[T1]{fontenc}
  \usepackage[utf8]{inputenc}
  \usepackage{textcomp} % provide euro and other symbols
\else % if luatex or xetex
  \usepackage{unicode-math}
  \defaultfontfeatures{Scale=MatchLowercase}
  \defaultfontfeatures[\rmfamily]{Ligatures=TeX,Scale=1}
\fi
% Use upquote if available, for straight quotes in verbatim environments
\IfFileExists{upquote.sty}{\usepackage{upquote}}{}
\IfFileExists{microtype.sty}{% use microtype if available
  \usepackage[]{microtype}
  \UseMicrotypeSet[protrusion]{basicmath} % disable protrusion for tt fonts
}{}
\makeatletter
\@ifundefined{KOMAClassName}{% if non-KOMA class
  \IfFileExists{parskip.sty}{%
    \usepackage{parskip}
  }{% else
    \setlength{\parindent}{0pt}
    \setlength{\parskip}{6pt plus 2pt minus 1pt}}
}{% if KOMA class
  \KOMAoptions{parskip=half}}
\makeatother
\usepackage{xcolor}
\IfFileExists{xurl.sty}{\usepackage{xurl}}{} % add URL line breaks if available
\IfFileExists{bookmark.sty}{\usepackage{bookmark}}{\usepackage{hyperref}}
\hypersetup{
  pdftitle={Final Report},
  pdfauthor={Iciar Fernandez Boyano, Jacob Gerlofs},
  hidelinks,
  pdfcreator={LaTeX via pandoc}}
\urlstyle{same} % disable monospaced font for URLs
\usepackage[margin=1in]{geometry}
\usepackage{color}
\usepackage{fancyvrb}
\newcommand{\VerbBar}{|}
\newcommand{\VERB}{\Verb[commandchars=\\\{\}]}
\DefineVerbatimEnvironment{Highlighting}{Verbatim}{commandchars=\\\{\}}
% Add ',fontsize=\small' for more characters per line
\usepackage{framed}
\definecolor{shadecolor}{RGB}{248,248,248}
\newenvironment{Shaded}{\begin{snugshade}}{\end{snugshade}}
\newcommand{\AlertTok}[1]{\textcolor[rgb]{0.94,0.16,0.16}{#1}}
\newcommand{\AnnotationTok}[1]{\textcolor[rgb]{0.56,0.35,0.01}{\textbf{\textit{#1}}}}
\newcommand{\AttributeTok}[1]{\textcolor[rgb]{0.77,0.63,0.00}{#1}}
\newcommand{\BaseNTok}[1]{\textcolor[rgb]{0.00,0.00,0.81}{#1}}
\newcommand{\BuiltInTok}[1]{#1}
\newcommand{\CharTok}[1]{\textcolor[rgb]{0.31,0.60,0.02}{#1}}
\newcommand{\CommentTok}[1]{\textcolor[rgb]{0.56,0.35,0.01}{\textit{#1}}}
\newcommand{\CommentVarTok}[1]{\textcolor[rgb]{0.56,0.35,0.01}{\textbf{\textit{#1}}}}
\newcommand{\ConstantTok}[1]{\textcolor[rgb]{0.00,0.00,0.00}{#1}}
\newcommand{\ControlFlowTok}[1]{\textcolor[rgb]{0.13,0.29,0.53}{\textbf{#1}}}
\newcommand{\DataTypeTok}[1]{\textcolor[rgb]{0.13,0.29,0.53}{#1}}
\newcommand{\DecValTok}[1]{\textcolor[rgb]{0.00,0.00,0.81}{#1}}
\newcommand{\DocumentationTok}[1]{\textcolor[rgb]{0.56,0.35,0.01}{\textbf{\textit{#1}}}}
\newcommand{\ErrorTok}[1]{\textcolor[rgb]{0.64,0.00,0.00}{\textbf{#1}}}
\newcommand{\ExtensionTok}[1]{#1}
\newcommand{\FloatTok}[1]{\textcolor[rgb]{0.00,0.00,0.81}{#1}}
\newcommand{\FunctionTok}[1]{\textcolor[rgb]{0.00,0.00,0.00}{#1}}
\newcommand{\ImportTok}[1]{#1}
\newcommand{\InformationTok}[1]{\textcolor[rgb]{0.56,0.35,0.01}{\textbf{\textit{#1}}}}
\newcommand{\KeywordTok}[1]{\textcolor[rgb]{0.13,0.29,0.53}{\textbf{#1}}}
\newcommand{\NormalTok}[1]{#1}
\newcommand{\OperatorTok}[1]{\textcolor[rgb]{0.81,0.36,0.00}{\textbf{#1}}}
\newcommand{\OtherTok}[1]{\textcolor[rgb]{0.56,0.35,0.01}{#1}}
\newcommand{\PreprocessorTok}[1]{\textcolor[rgb]{0.56,0.35,0.01}{\textit{#1}}}
\newcommand{\RegionMarkerTok}[1]{#1}
\newcommand{\SpecialCharTok}[1]{\textcolor[rgb]{0.00,0.00,0.00}{#1}}
\newcommand{\SpecialStringTok}[1]{\textcolor[rgb]{0.31,0.60,0.02}{#1}}
\newcommand{\StringTok}[1]{\textcolor[rgb]{0.31,0.60,0.02}{#1}}
\newcommand{\VariableTok}[1]{\textcolor[rgb]{0.00,0.00,0.00}{#1}}
\newcommand{\VerbatimStringTok}[1]{\textcolor[rgb]{0.31,0.60,0.02}{#1}}
\newcommand{\WarningTok}[1]{\textcolor[rgb]{0.56,0.35,0.01}{\textbf{\textit{#1}}}}
\usepackage{graphicx,grffile}
\makeatletter
\def\maxwidth{\ifdim\Gin@nat@width>\linewidth\linewidth\else\Gin@nat@width\fi}
\def\maxheight{\ifdim\Gin@nat@height>\textheight\textheight\else\Gin@nat@height\fi}
\makeatother
% Scale images if necessary, so that they will not overflow the page
% margins by default, and it is still possible to overwrite the defaults
% using explicit options in \includegraphics[width, height, ...]{}
\setkeys{Gin}{width=\maxwidth,height=\maxheight,keepaspectratio}
% Set default figure placement to htbp
\makeatletter
\def\fps@figure{htbp}
\makeatother
\setlength{\emergencystretch}{3em} % prevent overfull lines
\providecommand{\tightlist}{%
  \setlength{\itemsep}{0pt}\setlength{\parskip}{0pt}}
\setcounter{secnumdepth}{-\maxdimen} % remove section numbering

\title{Final Report}
\author{Iciar Fernandez Boyano, Jacob Gerlofs}
\date{}

\begin{document}
\maketitle

\hypertarget{the-mental-health-toll-of-graduate-school}{%
\section{The Mental Health Toll of Graduate
School}\label{the-mental-health-toll-of-graduate-school}}

\hypertarget{introduction}{%
\subsection{Introduction}\label{introduction}}

For the past five years, the iconic science journal Nature has launched
a survey for PhD students in STEM fields to share their experience in
graduate school, hoping to illuminate the goals, challenges, and sources
of satisfaction for doctoral students across seven continents. Last
year's survey collected data from over 6000 graduate students, which
constitutes the highest response rate in the survey's history. The full
data from the survey was made publicly available following publication
of an article discussing the results. It is interesting to note that the
survey was offered in English, Spanish, Chinese, French, and Portuguese
- open-form questions have not been translated to English if answered by
the participant in another language. Available materials include
anonymysed raw data, the questionnaire that was provided to PhD
students, and a presentation of the survey data.

In our project, we aim to investigate the relationship between these two
question areas (mental health \& feelings of harrasment/bullying) and
other variables that we hypothesise may be related to positive and/or
negative outcomes. For example, are those pursuing a degree far from
home more likely to suffer from anxiety and depression? Are instances of
harrasment and/or bullying male-biased? In an effort to shed some light
into the matter, we will study these questions in detail.

\hypertarget{research-question}{%
\subsection{Research question}\label{research-question}}

Rather than a single question, the many variables available as part of
our dataset has allowed us to investigate several relationships.

\begin{enumerate}
\def\labelenumi{\arabic{enumi}.}
\item
  \textbf{PhD Satisfaction.} We hypothesise that the level of
  satisfaction that a graduate student may feel with their decision to
  pursue a PhD program may be associated with (1) their university's
  long hours culture, (2) their work/life balance, and (3) their
  relationship with their supervisor.
\item
  \textbf{Suffering from anxiety or depression caused by PhD studies.}
  We are interested in seeing whether this variable is influenced by (1)
  the student's relationship with their supervisor, and (2) studying
  outside of their home country.
\item
  \textbf{Suffering from discrimination or harrassment}. We have
  investigated the relationship of this variable with (1) studying
  outside of your home country, and (2) the student's gender. \#\# Data
  and methods
\end{enumerate}

\hypertarget{data-description}{%
\subsubsection{Data Description}\label{data-description}}

According to the script with survey information that was provided, there
were a total of 65 questions. Not all questions were mandatory, and
there was a mix of single choice (yes/no), multiple choice (several
options) and free-form questions.

In the dataset, each row represents an individual who participated in
the survey, whereas each row represents a question. We have noticed some
redundancy in the dataset column that will require substantial cleanup
of the data as part of our next project milestone. For instance, Q12
(``What prompted you to study outside your country of upbringing?'') was
presented in the survey as a multiple choice question with 11 possible
answers (a-k), with the last one (k) being open-form (``If other, please
specify''). In the data frame, 11 rows correspond to Q12, each one
composed of 2 values: NA, and 1/11 possible answers. As such, the column
named Q12\_1 only contains NA values and answer ``(a) To study at a
specific university''; whereas Q12\_2 only contains NA values and answer
``(b) Lack of funding opportunities in my home country'', and so on. We
plan on combining columns Q12\_1:Q12\_11 into a single Q12 column using
dplyr::coalesce(), following the same rationale for other redundant
columns in the dataset. In addition, open-form questions such as (k) in
this specific example will be dropped due to the difficulty in analyzing
this, and the fact that they contain answers in different languages.

Due to this redundancy, the dimensions of the raw dataset when
downloaded are 6812 rows (participants) by 274 columns (questions),
whereas the actual survey only has 63 questions. Below is the complete
list of questions, which is a simplified version of the Word document
provided here, which includes all the possible answers for each
question. For simplicity, we have only included the question, its type,
and the category it belongs to within the survey.

\begin{verbatim}
## [1] 63  4
\end{verbatim}

Question\_No

Section

Question

Type

1

Questionnaire

Which, if any, of the following degrees are you currently studying for?

Single choice

2

Questionnaire

Which was the most important reason you decided to enrol in a PhD
programme?

Single choice

3

Questionnaire

Are you studying in the country you grew up in?

Single choice

4

Questionnaire

Where do you currently live?

Single choice

5

Questionnaire

Which country in Asia?

Single choice

6

Questionnaire

Which country in Australasia?

Single choice

7

Questionnaire

Which country in Africa?

Single choice

8

Questionnaire

Which country in Europe?

Single choice

9

Questionnaire

Which country in North or Central America?

Single choice

10

Questionnaire

Which country in South America?

Single choice

11

Questionnaire

What prompted you to study outside your country of upbringing?

Multiple choice

12

Questionnaire

Do you have a job alongside your studies?

Single choice

13

Questionnaire

What is your main reason for having a job?

Single choice

14

PhD Highs and Lows

What concerns you the most since you started your PhD?

Multiple choice

14a

PhD Highs and Lows

Is there anything else not mentioned that has concerned you since you
started your PhD?

Open

15

PhD Highs and Lows

Overall, what do you enjoy most about life as a PhD student?

Single choice

16

PhD Highs and Lows

How satisfied are you with your decision to pursue a PhD?

Scale

17

Satisfaction with your PhD experience

How satisfied are you with your PhD experience?

Scale

18

Satisfaction with your PhD experience

Since the very start of your graduate school experience, would you say
your level of satisfaction has:

Single choice

19

Satisfaction with your PhD experience

How satisfied are you with each of the following attributes or aspects
of your PhD?

Scale

20

Satisfaction with your PhD experience

To what extent does your PhD programme compare to your original
expectations?

Single choice

21

Your programme

On average, how many hours a week do you typically spend on your PhD
programme?

Single choice

22

Your programme

On average, how much one-on-one contact time do you spend with your
supervisor each week?

Single choice

23

Your programme

Overall, how would you describe the academic system, based on your PhD
experience so far?

Open

24

Your programme

To what extent do you agree or disagree with the following statements
regarding other faculty members or scientists in your department?

Scale

25

Your programme

Have you ever sought help for anxiety or depression caused by PhD study?

Single choice

26

Your programme

Did you seek help for anxiety or depression within your institution?

Single choice

27

Your programme

To what extent do you agree or disagree with the following statements?

Scale

28

Mental health and discrimination

Do you feel that you have experienced bullying in your PhD program?

Single choice

29

Mental health and discrimination

Who was the perpetrator(s)?

Multiple choice

29a

Mental health and discrimination

Do you feel able to speak out about your experiences of bullying without
personal repercussions?

Single choice

30

Mental health and discrimination

Do you feel that you have experienceddiscrimination or harassment in
your PhD program?

Single choice

31

Mental health and discrimination

Which of the following have you experienced?

Multiple choice

32

Future career plans

How much do you expect your PhD to improve your job prospects?

Scale

33

Future career plans

Which of the following sectors would you most like to work in (beyond a
postdoc) when you complete your degree?

Multiple choice and ranking

34

Future career plans

Please use the scale below to indicate how likely you are to pursue one
of these career paths upon completion of your programme.

Scale

35

Future career plans

If you're unlikely to pursue an academic research career, what are the
main reasons?

Multiple choice

36

Future career plans

What position do you most expect to occupy immediately after you
complete your degree?

Single choice

37

Future career plans

What type of career you are interested in pursuing after your graduate
degree?

Open

38

Career expectations

After completing your PhD, how long do you think it will take you to
find a permanent (non-trainee) position?

Single choice

39

Career expectations

How much more likely are you now to pursue a research career than when
you launched your PhD programme?

Single choice

40

Career expectations

What is the main reason why you are more likely to pursue a research
career?

Single choice

41

Career expectations

How did you arrive at your current career decision?

Multiple choice

42

Career support

How do you learn about available career opportunities that are beyond
academia?

Multiple choice

43

Career support

Which of the following 3 things would you say are the most difficult for
PhD students in your discipline?

Multiple choice

44

Career support

Which of the following would you say are the most difficult for PhD
students in the country where you are studying?

Multiple choice

45

Career support

Which of the following resources do you think PhD students need the most
in order to establish a satisfying career?

Multiple choice

46

Career support

How well is your programme preparing you to carry out each of the
following activities?

Scale

47

Career support

To what extent do you agree or disagree with the following statements?

Scale

48

Career support

Which, if any, of the following activities have you done to advance your
career?

Multiple choice

49

Career support

Which of the following social media networks have you used to build your
professional network?

Multiple choice

50

Reflection

What would you do differently right now if you were starting your
programme?

Multiple choice

51

Reflection

With the benefit of hindsight, what one thing do you know now which you
wish you'd known about when you started your PhD?

Open

52

Reflection

What is your age?

Single choice

53

Reflection

Are you\ldots{} (Gender)

Single choice

54

Reflection

Which of the following best describes you? (Ethnicity)

Multiple choice

55

Reflection

Do you have any caring responsibilities?

Multiple choice

56

Reflection

Thank you for taking part in the survey. Are there any more comments
you'd like to share with us?

Open

57

Thank you

Would you like to be entered into a prize draw for a chance to win GBP
£250? You can find prize draw terms and conditions here. Shift Learning
will be administering the incentive and a winner will be contacted
within 4 weeks of the survey closing date.

Single choice

58

Thank you

Nature may want to contact you again to ask for more information on the
subjects discussed in this survey, or to ask you specific questions
about your comments and answers. Are you happy to receive follow up
requests?

Single choice

59

Thank you

Springer Nature is keen to update PhD students with advice and
information about their programme and career options via a regular
newsletter. Would you like to be kept informed about this planned
service from Nature Careers?

Single choice

60

Thank you

Shift Learning carry out paid research in the education sector
throughout the year. Would you be happy to be contacted about relevant
future research opportunities?

Single choice

61

Thank you

Please fill in your contact details below.

Open

After careful consideration, we have only used certain variables for our
analysis. Below is a dataframe with these variables:

Variables

Type

Description

Gender

factor

Female (including trans female) / Male (including trans male) / Gender
queer

Studying in your home country

factor

Yes / No

Level of satisfaction with PhD

factor

1-5 Scale (1 - Very dissatisfied / 5 - Very satisfied)

Supervisor Relationship

factor

1-7 Scale (1 - Not at all satisfied / 7 - Extremely satisfied

Work Life Balance

factor

1-5 Scale (1 - Strongly disagree / 5 - Strongly agree

University Long Hours Culture

factor

1-5 Scale (1 - Strongly disagree / 5 - Strongly agree

Anxiety or Depression caused by PhD

factor

Yes / No / Prefer not to say

Experienced Bullying in PhD

factor

Yes / No / Prefer not to say

Experienced Discrimination or Harrassment in PhD

factor

Yes / No / Prefer not to say

\hypertarget{methods}{%
\subsubsection{Methods}\label{methods}}

Instead of linear regression, we have chosen logistic regression as the
statistical method to investigate the relationships that we have
outlined above in our ``Research Question'' section for our data.
Logistic regression is used when the dependent variable is categorical,
which is our case. Linear regression is not suitable for a
classification problem because it is unbounded.

\hypertarget{results}{%
\subsection{Results}\label{results}}

Prior to logistic regression, we performed some exploratory data
analysis in our data as part of
\href{https://github.com/STAT547-UBC-2019-20/group05/tree/master/docs/milestone-01}{milestone
1}.

\hypertarget{exploratory-data-analysis}{%
\subsubsection{Exploratory data
analysis}\label{exploratory-data-analysis}}

\hypertarget{general-demographics-of-graduate-students-in-the-survey}{%
\paragraph{General Demographics of Graduate Students in the
Survey}\label{general-demographics-of-graduate-students-in-the-survey}}

\includegraphics{here::here("images", "basic_demographics1.png")}

While the survey was available in several languages, which we assume was
an attempt to improve outreach, it is clear that the respondents origin
is highly biased to those located in Europe, closely followed by Asia
and North America.

\hypertarget{logistic-regression}{%
\subsubsection{Logistic regression}\label{logistic-regression}}

We then perform logistic regression of the relationships that we attempt
to investigate, described in the \textbf{research question} section.

\hypertarget{lr1-long-hours-culture-phd-satisfaction}{%
\paragraph{LR1: Long Hours Culture \textasciitilde{} PhD
Satisfaction}\label{lr1-long-hours-culture-phd-satisfaction}}

Interestingly, there isn't a clear trend in the relationship between
long hours culture \& satisfaction. It is true that those dissatisfied
with their decision (1 \& 2 in the x axis) seem to have a higher
probability of their university having a long hours culture, but aside
from that, there is no clear pattern.

\begin{Shaded}
\begin{Highlighting}[]
\NormalTok{knitr}\OperatorTok{::}\KeywordTok{include_graphics}\NormalTok{(here}\OperatorTok{::}\KeywordTok{here}\NormalTok{(}\StringTok{"images"}\NormalTok{, }\StringTok{"logisticregression.png"}\NormalTok{), }\DataTypeTok{auto_pdf =} \KeywordTok{getOption}\NormalTok{(}\StringTok{"knitr.graphics.auto_pdf"}\NormalTok{, }\OtherTok{FALSE}\NormalTok{), }
    \DataTypeTok{dpi =} \OtherTok{NULL}\NormalTok{)}
\end{Highlighting}
\end{Shaded}

\includegraphics[width=29.17in]{C:/Users/iciar/OneDrive/Desktop/School/STAT547/group05/images/logisticregression}

\hypertarget{lr2-supervisor-relationship-phd-satisfaction}{%
\paragraph{LR2: Supervisor Relationship \textasciitilde{} PhD
Satisfaction}\label{lr2-supervisor-relationship-phd-satisfaction}}

It appears that probability of satisfaction does increase as the
supervisor relationship goes from very dissatisfied (1) to very
satisfied (7). However, the trend is not as pronounced as one may expect
in this case either. Findings in the second graph match those that
result from logistic regression, and indeed confirm that students that
have a good relationship with their supervisor are more satisfied with
their decision to pursue a PhD than those that don't.

\begin{Shaded}
\begin{Highlighting}[]
\NormalTok{knitr}\OperatorTok{::}\KeywordTok{include_graphics}\NormalTok{(here}\OperatorTok{::}\KeywordTok{here}\NormalTok{(}\StringTok{"images"}\NormalTok{, }\StringTok{"logisticregression2.png"}\NormalTok{), }\DataTypeTok{auto_pdf =} \KeywordTok{getOption}\NormalTok{(}\StringTok{"knitr.graphics.auto_pdf"}\NormalTok{, }\OtherTok{FALSE}\NormalTok{), }
    \DataTypeTok{dpi =} \OtherTok{NULL}\NormalTok{)}
\end{Highlighting}
\end{Shaded}

\includegraphics[width=29.17in]{C:/Users/iciar/OneDrive/Desktop/School/STAT547/group05/images/logisticregression2}

\begin{Shaded}
\begin{Highlighting}[]
\NormalTok{knitr}\OperatorTok{::}\KeywordTok{include_graphics}\NormalTok{(here}\OperatorTok{::}\KeywordTok{here}\NormalTok{(}\StringTok{"images"}\NormalTok{, }\StringTok{"satisfaction_v_supervis_relationship.png"}\NormalTok{), }\DataTypeTok{auto_pdf =} \KeywordTok{getOption}\NormalTok{(}\StringTok{"knitr.graphics.auto_pdf"}\NormalTok{, }\OtherTok{FALSE}\NormalTok{), }
    \DataTypeTok{dpi =} \OtherTok{NULL}\NormalTok{)}
\end{Highlighting}
\end{Shaded}

\includegraphics[width=33.33in]{C:/Users/iciar/OneDrive/Desktop/School/STAT547/group05/images/satisfaction_v_supervis_relationship}

\hypertarget{lr3-worklife-balance-phd-satisfaction}{%
\paragraph{LR3: Work/Life Balance \textasciitilde{} PhD
Satisfaction}\label{lr3-worklife-balance-phd-satisfaction}}

Out of all of our analysis, this one shows one of the most pronounced
relationships - there is a strong trend between students that are more
satisfied with their decision to pursue a PhD and a probability of
having a good work/life balance, which stays true in our jitter graph.

\begin{Shaded}
\begin{Highlighting}[]
\NormalTok{knitr}\OperatorTok{::}\KeywordTok{include_graphics}\NormalTok{(here}\OperatorTok{::}\KeywordTok{here}\NormalTok{(}\StringTok{"images"}\NormalTok{, }\StringTok{"logisticregression5.png"}\NormalTok{), }\DataTypeTok{auto_pdf =} \KeywordTok{getOption}\NormalTok{(}\StringTok{"knitr.graphics.auto_pdf"}\NormalTok{, }\OtherTok{FALSE}\NormalTok{), }
    \DataTypeTok{dpi =} \OtherTok{NULL}\NormalTok{)}
\end{Highlighting}
\end{Shaded}

\includegraphics[width=29.17in]{C:/Users/iciar/OneDrive/Desktop/School/STAT547/group05/images/logisticregression5}

\begin{Shaded}
\begin{Highlighting}[]
\NormalTok{knitr}\OperatorTok{::}\KeywordTok{include_graphics}\NormalTok{(here}\OperatorTok{::}\KeywordTok{here}\NormalTok{(}\StringTok{"images"}\NormalTok{, }\StringTok{"satisfaction_v_work_life_bal.png"}\NormalTok{), }\DataTypeTok{auto_pdf =} \KeywordTok{getOption}\NormalTok{(}\StringTok{"knitr.graphics.auto_pdf"}\NormalTok{, }\OtherTok{FALSE}\NormalTok{), }
    \DataTypeTok{dpi =} \OtherTok{NULL}\NormalTok{)}
\end{Highlighting}
\end{Shaded}

\includegraphics[width=33.33in]{C:/Users/iciar/OneDrive/Desktop/School/STAT547/group05/images/satisfaction_v_work_life_bal}

\hypertarget{lr4-supervisor-relationship-anxiety-or-depression}{%
\paragraph{LR4: Supervisor Relationship \textasciitilde{} Anxiety or
Depression}\label{lr4-supervisor-relationship-anxiety-or-depression}}

We had a great interest in mental health markers in graduate school,
with anxiety or depression being the clearest to measure out of the
survey variables. There is a positive relationship between having a good
relationship with your supervisor and a low probability of seeking help
for anxiety or depression, which matches our hypothesis.

\begin{Shaded}
\begin{Highlighting}[]
\NormalTok{knitr}\OperatorTok{::}\KeywordTok{include_graphics}\NormalTok{(here}\OperatorTok{::}\KeywordTok{here}\NormalTok{(}\StringTok{"images"}\NormalTok{, }\StringTok{"logisticregression3.png"}\NormalTok{), }\DataTypeTok{auto_pdf =} \KeywordTok{getOption}\NormalTok{(}\StringTok{"knitr.graphics.auto_pdf"}\NormalTok{, }\OtherTok{FALSE}\NormalTok{), }
    \DataTypeTok{dpi =} \OtherTok{NULL}\NormalTok{)}
\end{Highlighting}
\end{Shaded}

\includegraphics[width=29.17in]{C:/Users/iciar/OneDrive/Desktop/School/STAT547/group05/images/logisticregression3}

\hypertarget{lr5-studying-outside-of-your-home-country-anxiety-or-depression}{%
\paragraph{LR5: Studying Outside of your Home Country \textasciitilde{}
Anxiety or
Depression}\label{lr5-studying-outside-of-your-home-country-anxiety-or-depression}}

Studying in your home country and seeking help for anxiety or depression
do not appear to be correlated when running a logistic regression.

\begin{Shaded}
\begin{Highlighting}[]
\NormalTok{knitr}\OperatorTok{::}\KeywordTok{include_graphics}\NormalTok{(here}\OperatorTok{::}\KeywordTok{here}\NormalTok{(}\StringTok{"images"}\NormalTok{, }\StringTok{"logisticregression4.png"}\NormalTok{), }\DataTypeTok{auto_pdf =} \KeywordTok{getOption}\NormalTok{(}\StringTok{"knitr.graphics.auto_pdf"}\NormalTok{, }\OtherTok{FALSE}\NormalTok{), }
    \DataTypeTok{dpi =} \OtherTok{NULL}\NormalTok{)}
\end{Highlighting}
\end{Shaded}

\includegraphics[width=29.17in]{C:/Users/iciar/OneDrive/Desktop/School/STAT547/group05/images/logisticregression4}

\hypertarget{lr6-studying-outside-of-your-home-country-discrimination-or-harrassment}{%
\paragraph{LR6: Studying Outside of your Home Country \textasciitilde{}
Discrimination or
Harrassment}\label{lr6-studying-outside-of-your-home-country-discrimination-or-harrassment}}

Although seemingly weak, there does appear to be a correlation between
study location and having experienced discrimination or harrassment.

\begin{Shaded}
\begin{Highlighting}[]
\NormalTok{knitr}\OperatorTok{::}\KeywordTok{include_graphics}\NormalTok{(here}\OperatorTok{::}\KeywordTok{here}\NormalTok{(}\StringTok{"images"}\NormalTok{, }\StringTok{"logisticregression6.png"}\NormalTok{), }\DataTypeTok{auto_pdf =} \KeywordTok{getOption}\NormalTok{(}\StringTok{"knitr.graphics.auto_pdf"}\NormalTok{, }\OtherTok{FALSE}\NormalTok{), }
    \DataTypeTok{dpi =} \OtherTok{NULL}\NormalTok{)}
\end{Highlighting}
\end{Shaded}

\includegraphics[width=29.17in]{C:/Users/iciar/OneDrive/Desktop/School/STAT547/group05/images/logisticregression6}

\hypertarget{lr7-gender-discrimination-or-harrassment}{%
\paragraph{LR7: Gender \textasciitilde{} Discrimination or
Harrassment}\label{lr7-gender-discrimination-or-harrassment}}

There are different correlations between each gender and discrimination
or harrassment.

\begin{Shaded}
\begin{Highlighting}[]
\NormalTok{knitr}\OperatorTok{::}\KeywordTok{include_graphics}\NormalTok{(here}\OperatorTok{::}\KeywordTok{here}\NormalTok{(}\StringTok{"images"}\NormalTok{, }\StringTok{"logisticregression7.png"}\NormalTok{), }\DataTypeTok{auto_pdf =} \KeywordTok{getOption}\NormalTok{(}\StringTok{"knitr.graphics.auto_pdf"}\NormalTok{, }\OtherTok{FALSE}\NormalTok{), }
    \DataTypeTok{dpi =} \OtherTok{NULL}\NormalTok{)}
\end{Highlighting}
\end{Shaded}

\includegraphics[width=29.17in]{C:/Users/iciar/OneDrive/Desktop/School/STAT547/group05/images/logisticregression7}

\hypertarget{discussion-and-conclusion}{%
\subsection{Discussion and Conclusion}\label{discussion-and-conclusion}}

While we have certainly found some relationships between our variables
of interest, these have proven weaker than originally expected. This
makes sense given the complexity of the variables - the level of
satisfaction that a student may feel with their decision to pursue a PhD
is probably due to the added effect of many variables rather than to a
single factor, meaning that a relationship between two variables alone
is an incomplete picture of everything that may be taken into account.
Nonetheless, it was still interesting to investigate these relationships
and work with this dataset. Unfortunately, it did prove quite
challenging to use a dataset with many categorical variables that
required a lot of fixing, converting answers to scales, and ultimately
running logistic regression rather than linear, as the latter is
unbounded and cannot be used for categorical variables.

\hypertarget{references}{%
\subsection{References}\label{references}}

Nature.com. 2020. Phds: The Tortuous Truth. {[}online{]} Available at:
\url{https://www.nature.com/articles/d41586-019-03459-7} {[}Accessed 17
March 2020{]}.

\end{document}
